\section{INTRODUCTION}

\IEEEPARstart{C}{OVID-19} is a newly identified disease that is very contagious and has been rapidly spreading across different countries around the world. The virus that was first identified in Wuhan has now infected more than 3.5 million people around the whole world and causes more than 245,000 deaths. Common symptoms from COVID-19 are fever, dry cough, but in more serious cases, patients can experience difficulty in breathing. As more people are infected, communities that have been in close contact with infected patients are getting tested for COVID-19. The test used to carry out the test for COVID-19 uses PCR(Polymerase Chain Reaction) test which could take several days for the test results to be available as the test samples are sent to a centralized lab for analysis and can be time-consuming. There is a limited number of supplies of PCR tests which is a bottleneck for testing to be efficient. Several alternative methods have been considered to test patients that are COVID-19 positive including a CT scan of the lungs. CT scans of the lungs are faster and easier to detect COVID-19 presence in patients. As the number of infected patients increases exponentially, it can be hard to provide testing scans for patients because of the limited number of doctors. It is recommended that Artificial Intelligence systems are used to analyze the CT scans of lung patients to determine the infected region of the lungs with COVID-19 and monitor the disease progression as well as to compensate for the high number of patients. Specifically, we propose using self-supervised deep learning to analyze and create a pixel-level segmentation of CT scan images of patients’ lungs to determine the infected area of the CT lung images that includes ground-glass opacities and consolidation. CNN \cite{ref29} will be an important technique to be used in image processing as CNN is able to capture useful features instead of handcrafting features to be used to evaluate on the segmentation of the CT lung images. Our \textit{key contribution} in this paper is to integrate self-supervision into an existing network to improve the performance of the original network. We will extend the work of InfNet as InfNet is one of the high performing model that includes CNN and several techniques to segment the ground-glass opacities and the consolidation area of the CT lung images. We will integrate self-supervised learning to InfNet to improve the performance of InfNet.