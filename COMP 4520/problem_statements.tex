\section{Problem Statements}
There are several limitations in the related works described. First, getting a high performance in deep neural networks requires an abundant amount of annotated samples. Performance can be drastically reduced if there are not enough data samples to compensate for the model’s complexity. Second, learning complex data distributions require a higher model complexity to be able to fit the distribution with better performance. The related works utilize semi-supervised learning to increase the number of data samples to achieve higher performance. As pixel-level segmentation on CT images is a complex task, pixel-level segmentation requires a high model complexity to fit the distribution. Unfortunately, there is a limited number of publicly available COVID-19 datasets especially in the form of pixel-level segmentation. The limited number of samples available greatly reduces the performance of modeling complex distribution for pixel-level segmentation of CT scans lung images. 

To solve the challenges, We propose a model and technique that utilizes self-supervised learning to mitigate the limited number of publicly available COVID-19 CT lung images samples to segment the infected regions of CT lung images. 
