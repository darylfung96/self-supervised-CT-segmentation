\documentclass[journal]{IEEEtran}
\usepackage[usenames]{color}
\usepackage{epsfig}
\usepackage{graphics}
\usepackage{caption}
\usepackage{amsmath}
\usepackage{amssymb}
\usepackage{multirow}
\usepackage{cite}
\usepackage{array}
\usepackage{pslatex} 
\usepackage{url}
\usepackage{lineno}
\usepackage{graphicx}  % Written by David Carlisle and Sebastian Rahtz
\usepackage{setspace}
\usepackage{tikz}
\usepackage{hhline}
\usepackage{mathtools}
\usepackage{caption}
\usepackage{xcolor}
\usepackage{algorithm}
\usepackage{algorithmic}
\usepackage{placeins}

\usepackage[letterpaper]{geometry}
\geometry{verbose,tmargin=0.7in,bmargin=0.7in,lmargin=0.65in,rmargin=0.65in}
\setlength{\headheight}{17pt}
%\captionsetup{labelfont={up},font=small}
\captionsetup[figure]{name={Fig.},labelsep=period,font=small}
\captionsetup[table]{name={TABLE},labelsep=period,font=small}

\usepackage{abstract}
\renewcommand{\absnamepos}{flushleft}
\setlength{\absleftindent}{0pt}
\setlength{\absrightindent}{0pt}


% If IEEEtran.cls has not been installed into the LaTeX system files,
% manually specify the path to it like:
% \documentclass[journal]{../sty/IEEEtran}


\usepackage{amsmath}   % From the American Mathematical Society
\hyphenation{}

% This is to insert comments
\newcommand{\memoA}[1]{{\textcolor{red}{\bf{$<$\emph{#1}$>$}}}}
\newcommand{\memoB}[1]{{\textcolor{magenta}{\bf{$<$\emph{#1}$>$}}}}
\newcommand{\memoC}[1]{{\textcolor{blue}{\bf{$<$\emph{#1}$>$}}}}
\newcommand{\memoD}[1]{{\textcolor{green}{\bf{$<$\emph{#1}$>$}}}}

% This is to automatically remove them
%\renewcommand{\memoA}[1]{}
%\renewcommand{\memoB}[1]{}
%\renewcommand{\memoC}[1]{}
%\renewcommand{\memoD}[1]{}
%\renewcommand{\efloatseparator}{\mbox{}}
%\renewcommand{\theposttbl}{\Roman{posttbl}}


%%%% Subfigures %%%%%%%%%%%%%%%%%%%%%%%%%%%%%%%

%\renewcommand\setlength{1cm}

%%%%% MC 16/05/06 THIS IS FOR VHDL LANGUAGE %%%%% 
\usepackage{listings}
\lstset{language=[AMS]VHDL}
\lstset{basicstyle=\verbatim@font} 

%%%%%%%%%%%%%%%%%%%%%%%%%%%%%%%%%%%%%%%%%%%%%%%%%%
% The paper headers
% correct bad hyphenation here
\pagenumbering{gobble} 

% The paper headers
\usepackage[english]{babel}
\usepackage[utf8]{inputenc}
\usepackage{fancyhdr}
\pagestyle{fancy}
\fancyhf{}
\lhead{\textcolor{violet}{\Large\textbf{IEEE EMB}}}
\rhead{\textcolor{violet}{\Large\textbf{\textsc{Med          }}} \thepage}
%\fancyfoot[CE,CO]{\leftmark}
\pagestyle{fancyplain}


\title{\textcolor{violet}{Self-supervised COVID-19 CT lung image segmentation with increasing mask complexity in-painting through multiple networks}}


\begin{document}

% \begin{flushright} 
% AAAAAAAAAAAAAAAAAAAAAAAAAAAAAAAAAAAAa
% \end{flushright}.	
% \begingroup
%\let\newpage\relax% Void the actions of \newpage
%	\maketitle\thispagestyle{fancy}   
%\endgroup

%
%
% author names and IEEE memberships
% note positions of commas and nonbreaking spaces ( ~ ) LaTeX will not break
% a structure at a ~ so this keeps an author's name from being broken across
% two lines.
% use \thanks{} to gain access to the first footnote area
% a separate \thanks must be used for each paragraph as LaTeX2e's \thanks
% was not built to handle multiple paragraphs
%\author{Michael~Shell,~\IEEEmembership{Member,~IEEE,}
%        John~Doe,~\IEEEmembership{Fellow,~OSA,}
%        and~Jane~Doe,~\IEEEmembership{Life~Fellow,~IEEE}% <-this % stops a space

\author{\textbf{Daryl Fung}\\
Department of Computer Science \\
7779598 \\
COMP4520 \\
\vspace*{0.1cm}

\small
University of Manitoba, MB, Canada\\

\normalsize
	
	%\thanks{Manuscript received January 20, 2002; revised November 18, 2002.
	%        This work was supported by the IEEE.}% <-this % stops a space
	%\thanks{M. Shell is with the Georgia Institute of Technology.}}
	%\thanks{This paragraph of the first footnote will contain the date on which you submitted your paper for review. It will also
}

% note the % following the last \IEEEmembership and also the first \thanks - 
% these prevent an unwanted space from occurring between the last author name
% and the end of the author line. i.e., if you had this:
% 
% \author{....lastname \thanks{...} \thanks{...} }
%                     ^------------^------------^----Do not want these spaces!
%
% a space would be appended to the last name and could cause every name on that
% line to be shifted left slightly. This is one of those "LaTeX things". For
% instance, "A\textbf{} \textbf{}B" will typeset as "A B" not "AB". If you want
% "AB" then you have to do: "A\textbf{}\textbf{}B"
% \thanks is no different in this regard, so shield the last } of each \thanks
% that ends a line with a % and do not let a space in before the next \thanks.
% Spaces after \IEEEmembership other than the last one are OK (and needed) as
% you are supposed to have spaces between the names. For what it is worth,
% this is a minor point as most people would not even notice if the said evil
% space somehow managed to creep in.
%





\twocolumn[
\begin{@twocolumnfalse}
    \maketitle\thispagestyle{fancy}   
%	\begin{abstract}
		\noindent
		\textcolor{violet}{\textbf{ABSTRACT}} COVID-19 is the new outbreak of a contagious disease that infects the lungs. Currently, no vaccines or antiviral medicines exist for COVID-19 as COVID-19 is a new infectious disease that was first discovered around December 2019. As COVID-19 is a very contagious disease, cases appear faster than the amount of test kit available. Currently, the most common testing used is PCR(Polymerase Chain Reaction) test. These test samples are sent to a centralized lab for analysis which would take several days for the test results to be available. Due to the exponential rate of infections, the limited amount of test kits, and the long wait time for the test results to be available, many infected patients are unable to get tested and receive treatments. An alternative approach to test for COVID-19 patients is through computerized tomography (CT) scan of the lungs. CT scan can drastically reduce the time taken for test results to be available and this could speed up the testing time as well as the limiting number of testing kits available. We will propose a deep learning architecture that can evaluate different segmentation of the lungs from CT images to detect if a patient is infected with COVID-19 so that we can reduce the amount of time taken to carry out testing to determine if patients are infected with COVID-19. We will extend the work of InfNet and integrate self-supervised learning into InfNet to determine if there is a performance improvement. \\
		\\
\noindent		
\textcolor{violet}{\textbf{INDEX TERMS}}  Deep Learning, REMOVE THIS: TODO: update abstract, add multi-seg figure, add severity score performance\\
\\
\textcolor{violet}{\textbf{IMPACT STATEMENT}} The authors should include here a significance statement of no more than 30 words. The statement should summarize the main findings of the research work reported in the manuscript.
\vspace{0.5cm}
		
%	\end{abstract}
\end{@twocolumnfalse}
]

%{\small \emph{\textbf{Index Terms}} -- \textbf{Say something}}


% Note that keywords are not normally used for peerreview papers.

% For peer review papers, you can put extra information on the cover
% page as needed:
% \begin{center} \bfseries EDICS Category: 3-BBND \end{center}
%
% For peerreview papers, inserts a page break and creates the second title.
% Will be ignored for other modes.
\IEEEpeerreviewmaketitle

\section{INTRODUCTION}

\IEEEPARstart{C}{OVID-19} is a newly identified disease that is very contagious and has been rapidly spreading across different countries around the world. The virus that was first identified in Wuhan has now infected more than 3.5 million people around the whole world and causes more than 245,000 deaths. Common symptoms from COVID-19 are fever, dry cough, but in more serious cases, patients can experience difficulty in breathing. As more people are infected, communities that have been in close contact with infected patients are getting tested for COVID-19. The test used to carry out the test for COVID-19 uses PCR(Polymerase Chain Reaction) test which could take several days for the test results to be available as the test samples are sent to a centralized lab for analysis and can be time-consuming. There is a limited number of supplies of PCR tests which is a bottleneck for testing to be efficient. Several alternative methods have been considered to test patients that are COVID-19 positive including a CT scan of the lungs. CT scans of the lungs are faster and easier to detect COVID-19 presence in patients. As the number of infected patients increases exponentially, it can be hard to provide testing scans for patients because of the limited number of doctors. It is recommended that Artificial Intelligence systems are used to analyze the CT scans of lung patients to determine the infected region of the lungs with COVID-19 and monitor the disease progression as well as to compensate for the high number of patients. Specifically, we propose using self-supervised deep learning to analyze and create a pixel-level segmentation of CT scan images of patients’ lungs to determine the infected area of the CT lung images that includes ground-glass opacities and consolidation. CNN \cite{ref29} will be an important technique to be used in image processing as CNN is able to capture useful features instead of handcrafting features to be used to evaluate on the segmentation of the CT lung images. Our \textit{key contribution} in this paper is to integrate self-supervision into an existing network to improve the performance of the original network. We will extend the work of InfNet as InfNet is one of the high performing model that includes CNN and several techniques to segment the ground-glass opacities and the consolidation area of the CT lung images. We will integrate self-supervised learning to InfNet to improve the performance of InfNet.
\section{Related works}

Several works have been proposed to create image segmentation for CT scan lung images of COVID-19 positive patients. They have demonstrated effective solutions using deep neural networks to accurately predict if a patient has COVID-19 positive or negative. 

A study has been conducted that uses supervised learning to train multiple models for different tasks where the study uses both classification and image segmentation tasks for COVID-19 detection through multi-tasks learning. The study uses Inception Residual Recurrent Neural Network (IRRCNN) for the classification of COVID-19 detection and uses Nabla-Net (NABLA-N) network for infected region segmentation for X-ray and CT images scan \cite{ref3}. Transfer learning is used to retrain the IRRCNN model with samples to differentiate between COVID-19 positive samples and negative samples in the classification phase. Mathematical Morphological approaches are implemented for selecting appropriate contours for chest region selection in the segmentation phase with NABLA-N network. Some classical imaging and adaptive threshold approaches are applied to extract the features to identify infected regions of COVID-19. They used a total number of 5,216 samples of which 3,875 samples are pneumonia and 1,341 samples are normal.

Another study \cite{ref4} introduces a supervised learning feature variation block and progressive atrious spatial pyramid pooling block using COVID-segNet, a high accuracy network that can create a segmentation of COVID-19 infection from chest CT images. The network consists of an Encoder and a Decoder with residual skip connection connecting the encoder and the decoder at their respective layer, following the architecture of UNET \cite{ref5}. Their main findings include the introduction of an FV block and a PASPP block. FV block consists of three branches - contrast enhancement branch, position sensitive branch, and identity branch. These branches can enable automatic change of parameters to display positions and boundaries of COVID-19. The PASPP block takes features extracted from the FV block to acquire semantic information with a variety of receptive fields. The dataset that they used consists of 21,658 labeled chest CT images, of which 861 CT images are confirmed COVID-19. 

The paper above however uses supervised learning and conducted the study with a good amount of data samples to train the network to achieve high performance. They obtained their dataset from hospitals through obtaining permission. We would like to create a network that does not require a much-labeled dataset to be able to achieve good performance. By doing this method, we could bring this network forward to detect new lung diseases when there is not many datasets available. Besides that, the paper is only able to recognize the presence of COVID-19 in a patient, but the papers could not quantify the severity of the disease. 

While there is a limited number of public data samples available for CT COVID-19 lung image segmentation, it will not be feasible to train a network to achieve high performance. As there are not many COVID CT dataset that contains the segmentation ground-truth, we train a network that contains the segmentation of the infected region so that the prediction results from our model are more intuitive and easily comprehensible. Several kinds of research that resolve this issue. One method is to use semi-supervised learning to mitigate the problem of having a low number of data samples to improve the performance of deep neural networks. Instead of having to manually annotate the data, semi-supervised learning utilizes the unlabeled data samples to aid in the training for the network.

Fan et al. \cite{ref2} used semi-supervised learning to enlarge the limited number of training samples for CT lung image segmentation. They developed a model called InfNet and semi-InfNet. The InfNet version of the model uses a fully supervised method to predict the segmentation of the CT images for ground-glass opacities and consolidations. The model outputs 4 images of the segmentation for the CT lung images that contain either ground-glass opacities or consolidations with different image sizes. The segmentation of the different image sizes is resized to the same size as the ground truth of the segmentation to compute the loss function. They also use an edge loss to guide the model to predict the boundary area of the segmentation. To improve InfNet, they use semi-supervised by progressively enlarging the training dataset with unlabeled data using a random sampling strategy. Specifically, they generate pseudo labels for unlabeled CT lung images. The advantage of using semi-supervised learning is that we can generate pseudo labels to increase the number of data samples. However, semi-supervised learning still requires to generate new examples through the use of unlabeled CT lung images before being able to undergo its learning procedure. This requires the use of unlabeled CT lung images to generate weakly labeled samples that are treated normally as labeled CT lung images to be fed into the network to train. This would be more time consuming as the network would have to first be trained on the labeled CT lung images, then evaluate the trained network on the unlabeled CT lung images to convert the unlabeled CT lung images into labeled CT lung images. After which the whole labeled CT lung images would be retrained again. This would take more than 3 times the time to train a supervised version of the network.

Another study \cite{ref6} uses Task-Based Feature Extraction Network (TFEN) and Covid-19 Identification Network (CIN).  They propose to use a task-specific feature extraction network that is tailored to CT lung images with three different classes: Healthy, pneumonia, and COVID-19 cases. They also mentioned that the dataset for COVID-19 is still limited and there is not enough high-quality dataset. They treat the task-specific feature extraction network as autoencoders and train the overall TFEN module to extract the relevant features from the CT images. Then, they use CIN to perform classification on the extracted features from the TFEN module. They can easily detect the abnormal regions and differentiate between them very accurately by making use of prior information even when a person contains limited CT images. This helped them develop a semi-supervised feature extraction network that allows obtaining the relevant prior information to perform the classification to mimic human behaviors. However, this study does not undergo segmentation of the CT lung images for better diagnosis of the CT lung images.

There is a study that predicts the severity score of COVID-19 on chest x-ray with deep learning \cite{ref7}. They use a DenseNet model from the TorchXRayVision library as DenseNet models have been shown to predict Pneumonia well. They use a pre-training step to train the feature extraction layers and a task prediction layer. The pre-training step was used to generate general representations of lungs and other Chest X-rays (CXRs) that they would have unable to achieve from the small set of COVID-19 images available. They use a network that outputs 18 outputs of a representation of the image, 4 outputs that are a hand-picked subset which contains the radiological findings (pneumonia, consolidation, lung opacity, and infiltration), and a lung opacity output. This study however did not use evaluate the segmentation of the infected region. They were only able to classify the CT lung images.

Lin et al. \cite{ref8} noticed that class imbalance encountered during the training of dense detectors tend to overwhelm the cross entropy loss function. The negative samples that are easily classified comprise the majority of the loss and have a huge influence on the gradient. Instead, they propose a loss function, focal loss, to reduce the weight of easy examples and focus more on hard negatives. Focal loss was able to improve the performance of the dense detector when the dataset trained on has class imbalance. 

SGD remains a popular optimizer to be used in training deep learning networks. There are improvement to SGD that includes AdaGrad \cite{ref9} or Adam \cite{ref10} which uses adaptive learning to optimize the weights of the deep learning network. However, hyperparameter tuning are costly to ensure the improvement of performance in deep learning networks with adaptive learning. Zhang et al. \cite{ref11} present Lookahead that is less sensitive to suboptimals hyperparameters and reduce the need for additional effort to tune the hyperparameter. Lookahead optimizer warps around SGD or Adam and is able to achieve fast convergence across multiple deep learning tasks with minimal computational overhead.

%Another study \cite{ref9} uses fine-grained segmentation networks (FGSN) to produce a dense segmentation map. Instead of using manually annotated labels, they created labels in a self-supervised manner. Features were extracted in certain layers in the network and clustered using k-means clustering. The cluster assignments are then used as supervision for the training. They showed that the learned representations can contain useful information for visual localization. The performance of the model improved by as much as 15% when trained with self-supervised learning instead of just classification.

\section{Problem Statements}
The related works on segmentation for COVID-19 was trained on an abundant amount of annotated data samples of CT lung images. Getting a high performance in deep neural networks requires an abundant amount of annotated samples. Performance can be drastically reduced if there are not enough data samples to compensate for the model’s complexity. Likewise, complex data distributions to learn require a higher model complexity to be able to fit the distribution with better performance. As pixel-level segmentation on CT images is a complex task, pixel-level segmentation requires a high model complexity to fit the distribution. Unfortunately, there is a limited number of publicly available COVID-19 dataset especially in the form of pixel-level segmentation. The limited number of samples available greatly reduce the performance of modeling complex distribution for pixel-level segmentation of CT scans lung images.

The related work does not also consider the severity of the lungs of patients as a result from COVID-19. We will propose a model and technique that utilizes self-supervised learning to mitigate the limited number of publicly available COVID-19 CT lung images samples as well as a method to calculate severity score of the segmented regions of CT lung images. 

\section{Methodology}


In this section, we will show the details of the self-supervised InfNet for imaging segmentation model including the network architecture, the data preprocessing steps, and the loss function. We will show how self-supervised InfNet helps to improve generalization and performance of the model while having a limited number of data samples. We will also show the extension of our data preprocessing steps which further improves the performance of our model.

Supervised InfNet (Lung Infection Segmentation Network) will be used as our baseline to compare without using any semi-supervised learning algorithm. This is to show that the self-supervised learning method improves the performance of the baseline supervised learning InfNet for imaging segmentation. We will extend our work on supervised InfNet by adding self-supervision method to it.

We will not change the structure of the InfNet model and use the default parameters as included in their GitHub code. There will be two different types of the InfNet model - single InfNet and multi InfNet. 

The single InfNet will create a single-labeled segmentation of the image for the infected region. The single InfNet predicts if the region is either ground-glass opacities or consolidations. It represents ground-glass opacities or consolidations as the same label. This means that the single InfNet will only predict the infected region without classifying them more specifically.  The CT lung image is first passed into the initial convolutional layers of the single InfNet to extract the features of the CT lung image. Then, the features generated from the convolutional layer are fed into the partial decoder module, reverse attention module, and the edge detection module. The edge detection module is to help the network with the detection of the boundaries of the segmentation. The reverse attention and the partial decoder generates the segmentation of the infection regions of the CT lung images.

The prediction from the single InfNet represents the infected region and will act as a prior to be fed into the multi InfNet. The prior will be concatenated with the original CT image to be fed into the multi InfNet network. The multi InfNet network will be used to predict multiple-labeled segmentation. The multiple-labeled segmentation includes predicting the background, ground-glass opacities, and consolidations for the infected region. The multiple-labeled segmentation model will give each of the labels a different value instead of grouping them as one as what the single-segmentation model does.

\subsection{Self-supervised InfNet for imaging segmentation}

We will propose using a self-supervised method to improve the performance of deep neural networks to create pixel-level segmentation for CT scans for lung images of COVID-19 patients. We will integrate self-supervised inpainting to pre-train our network. Since image inpainting is similarly related to image segmentation, we will integrate the pre-training steps as image inpainting for our image segmentation network. 

The original InfNet model would generate 5 different predictions: the edge segmentation prediction and the other 4 are segmentation of the infected regions but of different sizes. To utilize the ability of self-supervised method for InfNet segmentation, we generate masks to be fed into the InfNet model. The last convolution layer that outputs the prediction is not used for the self-supervised case. However, the last convolutional layer is replaced with a different convolutional layer to reconstruct the image and the edge appropriately. Everything else is kept the same as the InfNet architecture. This way the network will learn meaningful representations of the CT images and we can use these meaningful representations to learn the segmentation of the infected regions of the CT lung images. After learning the self-supervised features for InfNet, the training continues as normal similar to the InfNet algorithm. The training will start with the weights trained using the self-supervised inpainting method. The last layer will be changed to its original layer instead of the replaced convolutional layer. 

By learning features from image inpainting, the model can learn more features that are related to image segmentation. As creating masks can be a complex task for the network to learn to inpaint, the mask can either be too complex for the network to start learning or too simple to be able to learn good representations. We will be using a coach network that increases the complexity of the masking of the CT images throughout the training of the network. The mask created will initially be relatively simple, once the network can predict the inpainting of the CT images with good performance, the coach will increase the complexity of the masking to reduce the performance of the network, similar to how Generative Adversarial Network (GAN) works. The loss for the coach network is constructed from the loss of the image inpainting from the InfNet. The coach network and the InfNet both work together as a MinMax algorithm. The InfNet will try and minimize the loss to generate better image inpainting while the coach network will try to increase the loss of the image inpainting through generating more complex masks. In the beginning, the masks generated by the coach network will be less complex. Through the training of the coach network, as the InfNet gets better at predicting image inpainting, the coach network will generate more complex masks. The loss fuction for the coach network is:
\begin{equation}
L_{coach}(x) = 1 - L_{rec}(x\odot M)
\end{equation}

where $M = C(x)$ which is created by the coach network. A constraint is applied to this loss function because the coach network would just create a mask that masks all regions. After all, noa context information would be present for the network to learn and a maximum loss will be achieved. The constraint is:
\begin{equation}
\hat{B}(x) = B(x) - SORT(B(x))^{k|B(x)} 
\end{equation}
\begin{equation}
M = C(x) = \sigma (\alpha \hat{B}(x))
\end{equation}

The backbone, B, of the coach network has a similar network architecture with the model that inpaints the CT images. SORT(B(x)) sorts the features in descending order over the activation map. k represents the $k^{th}$ elements in the sorted list and k helps to control the fraction of the image to be erased. The region that has scores lesser than the $k^{th}$ element will be erased from the images. If k is 0.75 then 0.75 fraction of the images will not be erased. The score is scaled into a range of [0, 1] using a sigmoid activation function. We keep $\alpha = 1$ while training the coach network.
The illustration of the coach network can be seen in \ref{fig:coach-arch}.

\begin{figure*}
	\centering
	\small
	\includegraphics[width=\linewidth]{coach.png}
	\caption{The architecture of the coach network for self-supervised inpainting. }
	\label{fig:coach-arch}
\end{figure*}

After the self-supervision training is finished, the single segmentation InfNet would reuse the self-supervised single InfNet network weights to train normally on the segmentation of the CT lung images. Likewise, the multi InfNet network would reuse the weights that were trained during self-supervised multi InfNet training to train normally on the segmentation of the CT lung images.

The proposed self-supervised single-labeled segmentation InfNet network architecture can be seen in \ref{fig:inf-net_arch}. The left side of the figure is the original Single InfNet architecture and the right side of the figure is the self-supervised Single InfNet. The last layer for each output prediction is replaced with a different linear activation layer. The linear activation layer will re-create the original image that is covered by the masks. 

\begin{figure*}
	\centering
	\small
	\includegraphics[width=\linewidth]{self-super-inf-net.png}
	\caption{The architecture of our self-supervised InfNet model. }
	\label{fig:inf-net_arch}
\end{figure*}


The proposed self-supervised multi-labeled segmentation InfNet network architecture is shown in \ref{fig:multi-inf-net_arch}. The changes in the architecture for the multi-labeled segmentation InfNet are similar to the single-labeled segmentation InfNet where the last layer of the layer is replaced with a different linear activation layer to output the inpainting of the original image. 

\begin{figure*}
	\centering
	\includegraphics[width=\linewidth]{self-super-multi-inf-net.png}
	\caption{The architecture of our self-supervised multi segmentation InfNet model. Highlighted green block is the difference between the original multi InfNet and our self-supervised multi InfNet.}
	\label{fig:multi-inf-net_arch}
\end{figure*}

\begin{algorithm}
	\caption{Pseudo code for self-supervised with InfNet}
	\label{alg:self-inf-net}
	\begin{algorithmic}
		\STATE \textbf{Input:} $D_{labeled}$ = [($inputImage_1$, $G_{t1}$), ...]
		\FOR {each epoch}
		\FOR{each coach step}
		\STATE mask = M(x)
		\STATE maskedInput = $mask \odot inputImage$
		\STATE $ predictedImage =network(maskedInput), inputImage$
		\STATE $L_{rec} = CrossEntropy(predictedImage, inputImage)$
		\STATE $L_{coach}(x) = 1 - L_{rec}$
		\STATE update coach weights
		\ENDFOR
		\FOR {each network step}
		\STATE $P_{labeled} = Preprocess(D_{labeled})$
		\STATE $inpaintingOutput = network(P_{labeled})$
		\STATE $L_{rec} = CrossEntropy(InpaintingOutput, inputImage)$
		\STATE backpropogate and save network weights
		\ENDFOR
		\ENDFOR 
		
		
		\FOR {each batch of $D_{labeled}$:}
		\STATE $P_{labeled}$ = Preprocess ($D_{labeled}$)
		\STATE $trainLoss = train(P_{labeled})$
		\STATE Backpropagate train loss
		\STATE $testLoss = test(P_{labeled})$
		\STATE save model weights, \textit{w}.
		\ENDFOR
	\end{algorithmic}
\end{algorithm}

The output of the single segmentation InfNet will include the edge of the segmentation and four single-labeled segmentation of the infected region of the CT lung images with different sizes as shown in \ref{fig:supervised-inf-net_arch}. A loss will be calculated for each of the outputs from the single InfNet model. The first loss function is the loss edge, $L_{edge}$ which guides the model in representing better segmentation boundaries. The other loss function is the segmentation loss, ${L_{seg}}$. The segmentation loss combines both the loss of Intersection over Union (IoU) and the binary cross entropy loss. The segmentation loss equation for the single InfNet is as follow:
\begin{equation}
L_{seg} = L_{IoU} + \lambda L_{BCE}
\end{equation}

\begin{figure}
	\small
	\includegraphics[width=88mm]{supervised-inf-net.png}
	\caption{Architecture of the supervised InfNet.  }
	\label{fig:supervised-inf-net_arch}
\end{figure}

The $\lambda$ is set to 1 for this experiment. The segmentation loss is adapted to all of the ${S_i}$ predicted output where ${S_i}$ are created from $f_i$ such that $i={3,4,5}$. 

The total loss function for the single InfNet model is then:
\begin{equation}
L_{total} = L_{seg}(G_t, S_g) + L_{edge} + 	\sum_{i=3}^{5}L_{seg}(G_t, S_i)
\end{equation}

The summation of the loss functions are calculated from the output of the three convolutional layers. $G_t$ refers to the ground truth labels. $S_g$ is the output from the parallel partial decoder to match with the ground truth label.

As for the multiple segmentation infected region InfNet. We also use the default model and hyperparameters from the InfNet code. We will however train the network without using any unlabeled images to be used as a supervised version. The CT lung images and prior (infected region) for the CT lung images are concatenated together before being fed into the multiple segmentation InfNet. The prior is generated from the single segmentation InfNet. The prior would contain the area of the infected region. However, the prior does not contain the labels for ground-glass opacities and consolidations. It just shows the infected regions. The multiple segmentation InfNet will label the CT lung images with background, ground-glass opacities, and consolidations. The architecture for multiple segmentation InfNet can be seen in \ref{fig:multi-inf-net_arch}. The loss function for the multiple segmentation InfNet is as follow:
\begin{equation}
L_{bce} = \frac{1}{N}\sum_{i=1}^{N} y_i \cdot log(\hat{y_i}) + (1-y_i)\cdot log(1-\hat{y_i})
\end{equation}

The loss function for multiple segmentation InfNet uses the binary cross-entropy between the predicted segmentation and the ground truth segmentation.

In order to improve the performance of the model and to aid in the generalization, we determine to use self-supervised learning to learn good representations of the CT scan of lung images. Self-supervised learning generates auxiliary tasks from the labeled data samples. For instance, when undergoing data augmentation with rotation, we could train the network to predict if the images have been rotated 0 degrees, 90 degrees, 180 degrees to learn representations of the images. 


%determine the severity score of the lung regions through several methods. The severity score of the lungs can be determined by using CT severity score (CT-SS) \cite{ref11}. The score uses lung opacification for extension of the infections in the lungs. CT-SS is an adaptation from the method previously used in patients after severe-acute respiratory syndrome (SARS) \cite{ref10} to describe ground-glass opacity, interstitial opacity, and air trapping. The lungs will be divided into 20 regions, the posterior apical segment of upper left lobe was divided into apical and posterior segmental regions, the anteromedial basal segment of lower left lobe will be divided into anterior and basal segmental regions. For each region, there contain a system attributing scores of 0, 1, 2, either parenchymal opacification involves 0%, less than 50%, or equal to or more than 50%. The CT-SS will calculate the sum of individual scored regions. The final value of CT-SS can range from 0 to 40.


We will compare our method against the supervised %and semi-supervised \cite{ref13,ref14} 
\cite {ref13}models trained on COVID-19 dataset. For comparing supervised learning, we will compare against the paper \cite{ref13}. We will train and follow using the same network structure but change from supervised learning to self-supervised learning and compare the performance between supervised and self-supervised.

%When comparing with the semi-supervised model, we determine that our model is successful if our model is able to reach close to or better than the performance of the semi-supervised model as semi-supervised model is able to obtain a higher amount of data samples by looking at both unannotated and annotated data samples while self-supervised model only have access to the annotated labels. A self-supervised learning method will create its own training annotated labels without any manual human labelling and trained without any unlabeled data samples. We will compare our method’s performance against InfNet \cite{ref14} which uses semi-supervised learning by generating pseudo labels from randomly selected unlabeled CT images.

We will use this approach to determine if self-supervised learning can be a useful task to help InfNet improve its performance in segmenting the ground-glass opacities or consolidation around the infected region of the CT lung images.

%Our method will be novel compare to the other methods mentioned as our method will be integrating both the segmentation of the CT lung images as well as the calculation of the severity score through caluclation of the segmented infected lung areas.



\iffalse
The pseudo-code for the training of baseline model is relatively straightforward as can be seen in \ref{alg:baseline}.
\begin{algorithm}
	\caption{Pseudo code for InfNet}
	\label{alg:baseline}
	\begin{algorithmic}
		\STATE \textbf{Input:} Train data $D_{labeled}$,  test data $D_{t-labeled}$ ground truth $G_t$,
		\FOR {each batch of $D_{labeled}$:}
		\STATE $P_{labeled}$ = Preprocess $D_{labeled}$
		\STATE $L_{seg}(G_t, S_g), L_{edge}, L_{seg}(G_t, S_3), L_{seg}(G_t, S_4), L_{seg}(G_t, S_5)	$  = train(\textit{M},  $P_{labeled}$)
		\STATE $L_{traintotal} = L_{seg}(G_t, S_g) + L_{edge} + L_{seg}(G_t, S_3) + L_{seg}(G_t, S_4) + L_{seg}(G_t, S_5)$
		\STATE plot($L_{traintotal}$)
		\STATE = test(\textit{M}, $D_{t-labeled}$)
		\STATE plot(test loss)
		\STATE save model weights, \textit{w}.
		\ENDFOR
	\end{algorithmic}
\end{algorithm}
\fi 
\section{Experiments}

\begin{table}[!h]
	\centering
	\begin{tabular}{|c||c|c|c|c|} \hline
		Data split & Source & Segmented & Images & Patients \\\hline
		Training & \vtop{\hbox{\strut Med-Seg}\hbox{\strut ICTCF}}&
		\vtop{\hbox{\strut Yes}\hbox{\strut No}} & 
		\vtop{\hbox{\strut 698}\hbox{\strut 6654}}&
		\vtop{\hbox{\strut 39}\hbox{\strut 1338}}\\\hline
		Validation & Med-Seg & Yes & 114 & 35 \\\hline
		Testing & Med-Seg & Yes & 117 & 35 \\\hline
	\end{tabular}
	\caption{This table shows the data distribution between the datasets that we use to evaluate our model on. Med-Seg refers to the COVID-19 CT Segmentation data set and ICTCF refers to the ICTCF data set.}
	\label{tab:dataset}
\end{table}

\begin{table*}[!h]
	\centering
	\begin{tabular}{| c | c || c c c c c ||}
		\hline
		Methods & & F1 & IoU & Recall & Precision & AUC \\ \hline
		Single SInfNet &  Mean & \textbf{0.39} & \textbf{0.29} & \textbf{0.83} & 0.33 & 0.9605 \\ \cline{2-7}
		& Error & $\pm$ 0.059 & $\pm$ 0.053 & $\pm$ \textbf{0.069} & $\pm$0.057  & $\pm$0.032 \\ \hline
		Single SInfNet + data aug(0.4) &  Mean & 0.38 & 0.27 & 0.79 & \textbf{0.34} & 0.9697 \\ \cline{2-7}
		& Error & $\pm$0.054  & $\pm$0.045  &$\pm$0.071 &$\pm$0.055 &$\pm$0.017 \\ \hline
		Single SInfNet + data aug (0.5) &  Mean & 0.37 & 0.26 & 0.81 & 0.32 & 0.9658 \\ \cline{2-7}
		& Error &$\pm$0.054 &$\pm$0.045 &$\pm$0.072 &$\pm$0.050 &  $\pm$0.021  \\ \hline \hline
		Single Self-SInfNet &  Mean & 0.38 & 0.27 & 0.75 & 0.33 & 0.9742  \\ \cline{2-7}
		& Error & $\pm$0.056 & $\pm$0.049 &$\pm$0.077  & $\pm$0.053 & $\pm$  0.010 \\ \hline
		Single Self-SInfNet + data aug &  Mean & 0.30 & 0.20 & 0.72 & 0.28 &  \textbf{0.9785} \\ \cline{2-7}
		& Error & $\pm$ \textbf{0.050}  & $\pm$  \textbf{0.039} & $\pm$ 0.085 & $\pm$\textbf{0.045} & $\pm$ \textbf{0.006}  \\ \hline
	\end{tabular}
	\caption{Quantitative result for comparison between Single segmentation InfNet and self-supervised single segmentation InfNet in the test set.}
	\label{tab:single}
\end{table*}

\begin{table*}[!h]
	\centering
	\small
	\begin{tabular}{| c | c || c c c c || c c c c |}
		\hline
		& &\multicolumn{4}{c||}{Ground-Glass Opacity} & \multicolumn{4}{c|}{Consolidation}\\ \cline{3-10}
		Methods & & F1 & IoU & Recall & Precision & F1 & IoU & Recall & Precision \\\hline
		SInfNet & Mean & 0.38 & 0.27 & 0.58 & 0.41 & 0.29 & 0.22 & 0.61 & 0.31  \\ \cline{2-10}
		& Error & $\pm$0.054 & $\pm$0.042 & $\pm$0.065 & $\pm$0.058 & $\pm$0.078 & $\pm$0.068 & $\pm$0.099 & $\pm$0.084  \\ \hline \hline
		
		\vtop{\hbox{\strut SInfNet+}\hbox{\strut data aug(0.4)}} & Mean & 0.34 & 0.24 & 0.54 & 0.39 & 0.23 & 0.17 & 0.51 & 0.29   \\ \cline{2-10}
		& Error & $\pm$0.056 & $\pm$0.044 & $\pm$0.072 & $\pm$0.057 & $\pm$0.067 & $\pm$0.056 & $\pm$0.104 & $\pm$0.082 \\ \hline \hline
		
		\vtop{\hbox{\strut SInfNet+}\hbox{\strut data aug(0.5)}} & Mean & 0.34 & 0.24 & 0.53 & 0.38 & 0.24 & 0.18 & 0.47 & 0.31  \\ \cline{2-10}
		& Error & $\pm$0.054 & $\pm$0.042 & $\pm$0.068 & $\pm$0.057 & $\pm$0.075 & $\pm$0.064 & $\pm$0.114 & $\pm$0.084 \\ \hline \hline
		
		\vtop{\hbox{\strut SSInfNet}\hbox{\strut }} & Mean & 0.34 & 0.24 & 0.54 & 0.39 &  0.23 & 0.17 & 0.51 & 0.29 \\ \cline{2-10}
		& Error & $\pm$0.056 & $\pm$0.044 & $\pm$0.072 & $\pm$0.057 & $\pm$0.067 & $\pm$0.056 & $\pm$0.104 & $\pm$0.082 \\ \hline \hline
		
		\vtop{\hbox{\strut SSInfNet+}\hbox{\strut data aug}} & Mean & 0.34 & 0.24 & 0.54 & 0.39 & 0.23 & 0.17 & 0.51 & 0.29\\ \cline{2-10}
		& Error & $\pm$0.056 & $\pm$0.044 & $\pm$0.072 & $\pm$0.057 & $\pm$0.067 & $\pm$0.056 & $\pm$0.104 & $\pm$0.082 \\ \hline \hline \hline
		
		
		& &\multicolumn{4}{c||}{Background} & \multicolumn{4}{c|}{Overall}\\ \cline{3-10}
		Methods & & F1 & IoU & Recall & Precision & F1 & IoU & Recall & Precision \\\hline
		SInfNet & Mean & 1.0 & 0.99 & 0.99 & 1.0 & 0.55 & 0.5 & 0.73 & 0.57   \\ \cline{2-10}
		& Error & $\pm$0.002 & $\pm$0.003 & $\pm$0.002 & $\pm$0.002 & $\pm$0.044 & $\pm$0.038 & $\pm$0.055 & $\pm$0.048 \\ \hline
		\vtop{\hbox{\strut SInfNet+}\hbox{\strut data aug(0.4)}} & Mean & 0.99 & 0.99 & 0.99 & 1.0 & 0.52 & 0.47 & 0.68 & 0.56   \\ \cline{2-10}
		& Error &$\pm$0.002 & $\pm$0.003 & $\pm$0.002 & $\pm$0.002 & $\pm$0.042 & $\pm$0.034 & $\pm$0.059 & $\pm$0.047  \\ \hline \hline
		\vtop{\hbox{\strut SInfNet+}\hbox{\strut data aug(0.5)}} & Mean &0.99 & 0.99 & 0.99 & 1.0 & 0.52 & 0.47 & 0.67 & 0.56 \\ \cline{2-10}
		& Error & $\pm$0.002 & $\pm$0.004 & $\pm$0.002 & $\pm$0.002 & $\pm$0.043 & $\pm$0.036 & $\pm$0.061 & $\pm$0.048\\ \hline \hline
		\vtop{\hbox{\strut SSInfNet}\hbox{\strut }} & Mean & 0.99 & 0.99 & 0.99 & 1.0 & 0.52 & 0.47 & 0.68 & 0.56 \\ \cline{2-10}
		& Error & $\pm$0.002 & $\pm$0.003 & $\pm$0.002 & $\pm$0.002 & $\pm$0.042 & $\pm$0.034 & $\pm$0.059 & $\pm$0.047\\ \hline \hline
		\vtop{\hbox{\strut SSInfNet+}\hbox{\strut data aug}} & Mean & 0.99 & 0.99 & 0.99 & 1.0 & 0.52 & 0.47 & 0.68 & 0.56\\ \cline{2-10}
		& Error & $\pm$0.002 & $\pm$0.003 & $\pm$0.002 & $\pm$0.002 & $\pm$0.042 & $\pm$0.034 & $\pm$0.059 & $\pm$0.047 \\ \hline \hline \hline
		
	\end{tabular}
	\caption{Quantitative result of Ground-glass Opacities \& Consolidation on the test data set. Prior is obtained from the single segmentation InfNet}
	\label{tab:multi-weakprior}
\end{table*}

\begin{table*}[!h]
	\centering
	\small
	\begin{tabular}{| c | c || c c c c || c c c c |}
		\hline
		& &\multicolumn{4}{c||}{Ground-Glass Opacity} & \multicolumn{4}{c|}{Consolidation}\\ \cline{3-10}
		Methods & & Dice & Jac & Recall & Precision & Dice & Jac & Recall & Precision \\\hline
		SInfNet & Mean & 0.87 & 0.81 & 0.87 & 0.9 & 0.47 & 0.39 & 0.68 & 0.55\\ \cline{2-10}
		& Error & $\pm$0.041 & $\pm$0.049 & $\pm$0.043 & $\pm$0.042 & $\pm$0.102 & $\pm$0.093 & $\pm$0.103 & $\pm$0.111 \\ \hline \hline
		
		\vtop{\hbox{\strut SInfNet+}\hbox{\strut data aug(0.4)}} & Mean &0.86 & 0.81 & 0.87 & 0.91 & 0.58 & 0.47 & 0.64 & 0.74 \\ \cline{2-10}
		& Error & $\pm$0.048 & $\pm$0.058 & $\pm$0.052 & $\pm$0.04 & $\pm$0.096 & $\pm$0.092 & $\pm$0.108 & $\pm$0.095  \\ \hline \hline
		
		\vtop{\hbox{\strut SInfNet+}\hbox{\strut data aug(0.5)}} & Mean &0.86 & 0.81 & 0.88 & 0.9& 0.53 & 0.44 & 0.62 & 0.69   \\ \cline{2-10}
		& Error &$\pm$0.045 & $\pm$0.055 & $\pm$0.05 & $\pm$0.042 & $\pm$0.108 & $\pm$0.099 & $\pm$0.118 & $\pm$0.108  \\ \hline \hline

		SSInfNet & Mean & 0.86 & 0.8 & 0.87 & 0.89 & 0.55 & 0.46 & 0.67 & 0.68  \\ \cline{2-10}
		& Error & $\pm$0.041 & $\pm$0.051 & $\pm$0.044 & $\pm$0.039 & $\pm$0.106 & $\pm$0.101 & $\pm$0.116 & $\pm$0.108 \\ \hline \hline
		
		\vtop{\hbox{\strut SSInfNet+}\hbox{\strut data aug}}& Mean & 0.82 & 0.74 & 0.82 & 0.86 & 0.27 & 0.22 & 0.61 & 0.31   \\ \cline{2-10}
		& Error & $\pm$0.046 & $\pm$0.053 & $\pm$0.048 & $\pm$0.04 & $\pm$0.077 & $\pm$0.067 & $\pm$0.117 & $\pm$0.088\\ \hline \hline \hline
		
		
		& &\multicolumn{4}{c||}{Background} & \multicolumn{4}{c|}{Overall}\\ \cline{3-10}
		Methods & & F1 & IoU & Recall & Precision & F1 & IoU & Recall & Precision \\\hline
		SInfNet & Mean & 1.0 & 1.0 & 1.0 & 1.0 & 0.78 & 0.73 & 0.85 & 0.82 \\ \cline{2-10}
		& Error &$\pm$0.0 & $\pm$0.0 & $\pm$0.0 & $\pm$0.0 & $\pm$0.047 & $\pm$0.048 & $\pm$0.049 & $\pm$0.051 \\ \hline \hline
		\vtop{\hbox{\strut SInfNet+}\hbox{\strut data aug(0.4)}}  & Mean &1.0 & 1.0 & 1.0 & 1.0 & 0.81 & 0.76 & 0.84 & 0.88  \\ \cline{2-10}
		& Error & $\pm$0.0 & $\pm$0.0 & $\pm$0.0 & $\pm$0.0 & $\pm$0.048 & $\pm$0.05 & $\pm$0.053 & $\pm$0.045 \\ \hline \hline
		\vtop{\hbox{\strut SInfNet+}\hbox{\strut data aug(0.5)}}  & Mean & 1.0 & 1.0 & 1.0 & 1.0 & 0.8 & 0.75 & 0.83 & 0.86 \\ \cline{2-10}
		& Error & $\pm$0.0 & $\pm$0.0 & $\pm$0.0 & $\pm$0.0 & $\pm$0.051 & $\pm$0.051 & $\pm$0.056 & $\pm$0.05 \\ \hline \hline
		SSInfNet & Mean &1.0 & 1.0 & 1.0 & 1.0&0.8 & 0.75 & 0.85 & 0.86 \\ \cline{2-10}
		& Error &$\pm$0.0 & $\pm$0.0 & $\pm$0.0 & $\pm$0.0 & $\pm$0.049 & $\pm$0.05 & $\pm$0.054 & $\pm$0.049 \\ \hline \hline
		\vtop{\hbox{\strut SSInfNet+}\hbox{\strut data aug}} & Mean & 1.0 & 1.0 & 1.0 & 1.0 & 0.69 & 0.65 & 0.81 & 0.72 \\ \cline{2-10}
		& Error &$\pm$0.0 & $\pm$0.0 & $\pm$0.0 & $\pm$0.0 & $\pm$0.041 & $\pm$0.04 & $\pm$0.055 & $\pm$0.043 \\ \hline \hline
		
	\end{tabular}
	\caption{Quantitative result of Ground-glass Opacities \& Consolidation on the test data set. Prior is obtained from the test set.}
	\label{tab:multi-strongprior}
\end{table*}


\subsection{Datasets}
The dataset that we will be using is an integrative resource of chest computed tomography images and clinical features of patients with COVID-19 pneumonia (ICTCF) \cite{ref23} which contains the severity score for each CT lung image and CT lung images from medical segmentation website \cite{ref26}. 

ICTCF contains 127 types of clinical features and laboratory confirmed cases of COVID-19 from 1170 patients including the severity for the CT lung images. However, ICTCF dataset does not contain the segmentation labels for the ground-glass opacities and the consolidation in the CT lung images. In total, there are 6654 of CT lung images in ICTCF dataset. Originally, there were 1521 patients. However, some of the patients are missing CT lung images. We remove these patients that are missing CT lung images. After preprocessing the patients, the dataset was left with 1338 patients that contains CT lung images. The dataset can be found here: http://ictcf.biocuckoo.cn/. 

As for the medical segmentation dataset, they contain ground truth label for the segmentation for ground-glass opacities and consolidation of the CT lung images but does not contain the severity score for the CT Lung images. The total amount of CT lung images contain in medical segmentation dataset is 932 CT lung images. We randomly assign the CT lung images into training set, validation set, and testing set of which the training set contains 698 CT lung images, the validation set contains 114 CT lung images, and the testing set contains 117 CT lung images. 

The assignment of the dataset can be seen in \ref{tab:dataset}.


\begin{figure}
	\centering
	\small
	\includegraphics[width=\linewidth]{data_aug.png}
	\caption{Example of data augmentation on the CT lung images.}
	\label{fig:data_aug}
\end{figure}

\subsection{Experimental Settings}
During the self-supervised image inpainting stage, we train the network for 2000 epochs. The network is trained for the first 200 epochs before we train the coach network for 200 epochs which increases the complexity of the masks generated. After that, we alternate in between training the self-supervised image inpainting and the coach network with 100 epochs in between. For every alternating between the training of the self-supervised image inpainting and the coach network, we set the learning rate to 0.1 at the start of the epoch, we set the learning rate to 0.01 at 40th epoch, we set the learing rate to 0.001 at 80th eochs, and 0.0001 at the 90th epoch.  We use SGD as the optimizer for the self-supervised image inpainting.  We set the momentum to 0.9 and the weight decay to 0.0005. As for the optimizer for the coach network, we use Adam optimizer with learning rate of 0.00001.

For the Single InfNet, we train the network for 500 epochs. We use Adam as the optimizer with learning rate of 0.0001. 

For the Multi InfNet, we train the network for 500 epochs. We use SGD as the optimizer. The momentum is set as 0.7 and the learning rate is set as 0.01.

For the severity score calculation, there are several different labels obtained from ICTCF dataset for severity, The different severity are: \textit{Regular, Mild, Control, Severe, Critically ill.}  The assign the different labels with score ranging from 0 to 2. \textit{Regular, Mild, and Control} are assign having a severity score of 0. \textit{Severe} is assign having a severity score of 1. \textit{Critically ill} is assign having a severity score of 2. We use the metrics provided by sklearn to caclulate the F1 Score, Precision Score, and Recall Score for the severity score prediction. We use 'micro' as the averaging for the scores as provided by sklearn.

\subsection{Data Augmentation}
We used data augmentation to increase our data samples size. The data agumentation that we used includes \textit{vertical flipping, horizontal flipping, random crop, and random cutout}. For the random cutout percentage, we experimented that 0.5 cDuring tutout of the CT lung images yield higher performance than the rest of the value. This is because entropy at 0.5 is the highest which could increase more variability of the images. Examples of the data augmentation can be seen in figure \ref{fig:data_aug}.
The left column is the original CT lung images while the right column is the augmented CT lung images. The first row involves random cropping and random cutout. The second row involves random cropping and random cutout. The third row involves random cropping and vertical flipping. The random cutout involves patching the image with colors of the same value of rgb. For instance, if the value of r is 10, then the value of g and b are also 10. If the value of r is 50, then the value of g and b are also 50.


\begin{figure*}
	\includegraphics[width=\linewidth]{comparison_single.png}
	\caption{Comparison of single segmentation between different networks.}
	\label{fig:single-comparison}
\end{figure*}
\begin{figure*}
	\includegraphics[width=\linewidth]{comparison_multi_weakprior.png}
	\caption{Comparison of multi segmentation between different networks with prior generated from single InfNet.}
	\label{fig:multi-weakprior-comparison}
\end{figure*}
\begin{figure*}
	\includegraphics[width=\linewidth]{comparison_multi_strongprior.png}
	\caption{Comparison of multi segmentation between different networks with prior from Test Set.}
	\label{fig:multi-strongprior-comparison}
\end{figure*}

\begin{table}
	\begin{tabular}{|c||c|c|c|}
	\hline
	Method & F1 Score & Precision & Recall \\ \hline
	SInfNet & 0.28 & 0.28 & 0.28 \\ \hline
	SInfNet+data aug(0.4) & 0.61 & 0.61 & 0.61 \\ \hline
	SInfNet+data aug(0.5) & 0.42 & 0.42 & 0.42 \\ \hline
	Self SInfNet & 0.55 & 0.55 & 0.55 \\ \hline
	Self SInfNet+data aug & \textbf{0.62} & \textbf{0.62} & \textbf{0.62} \\ \hline
	\end{tabular}
	\caption{Table shows the result of severity score prediction on CT lung images using segmentation.}
	\label{tab:severity}
\end{table}

\section{Results}
In this section, we will show the results of our experiments obtained. We will divide this section into two different subsections: Result for self-supervised InfNet and result for estimation of severity score.

\subsection{Result for self-supervised InfNet}
 The result for our comparison between the baseline InfNet model and our self-supervised model can be seen in \ref{tab:single}, \ref{tab:multi-weakprior}, and \ref{tab:multi-strongprior}. The table is plotted with  several metrics: dice, jaccard, sensitvity, specificity, and mean absolute error (MAE). 
 
 For the table that contains mean and error, the mean are calculated as:
 \begin{equation}
mean =  \frac{\sum_{i=1}^{N}Metric(\hat{y_i}, y_i)}{N}
 \end{equation}
 Where Metric refers to either \textit{Dice, Jaccard, Sensitivity, Specificity, or mean absolute error (MAE).} N refers to the number of test data samples.
 The error is:
 \begin{equation}
error =  SE x 1.96
 \end{equation}
 where SE is the standard error of the test data samples for the metric multiplied by 1.96.
Note that Mean $\pm$Error is the 95\% confidence interval.
 
 We show several tables for our comparisons. \ref{tab:single} shows the result for the single segmentation InfNet. The single segmentation InfNet does not segment between ground-glass opacities or consolidation. The single segmentation will segment and represent all infected region as one. We can see that self-supervise can improve on the generalisation and consistency on predicting on the different CT lung images as they perform the best in terms of the error range. Even though the baseline single SInfNet performance have better mean values for dice, jaccard, and sensitivity, the self-supervised approach helps to create robustness and consistency in the model itself to better handle outliers. We can see the results of the single segmentation in \ref{fig:single-comparison}. We can see that the baseline single SInfNet overestimated the infected region of an outlier in the segmentation result in the figure in the last row. The baseline single SInfNet even with added data augmentation predicted some infected region in the CT lung images when the ground truth does not contain any infected region. The self-supervised SInfNet did a better job at predicting outlier's where its prediction is more closely related to the ground truth than the baseline single SInfNet.
 
 \ref{tab:multi-weakprior} shows the result for the comparison between multiple segmentation InfNet. As the multiple segmentation InfNet requires a CT lung image concatenate with a prior as input where the prior is the segmentation of the infected region of the CT lung without considering the location of ground-glass opacities or consolidation. The prior represents the infected region as a whole. For the result of this table, the prior is obtained from the single segmentation InfNet. Then the prior is fed into the multiple segmentation InfNet to obtain the result. We can see from the table that the self-supervised multi SInfNet does a better job at predicting multiple segmentation than the the baseline multi SInfNet even with when the baseline multi SInfNet has been added with data augmentation. Self-supervised helps to prevent loss of performance when the  prediction is needed to be transferred from one network to another network. In our case, the prediction is transferred from the single self-SInfNet as prior to the multi self-SInfNet.  Self-supervised also creates a more consistent and robust network to outliers. We can see the segmentation result in \ref{fig:multi-weakprior-comparison}. The self-supervised with data augmentation does not seem to improve the performance on the self-supervised multi InfNet. However, it does reduce the difference in the error variation between different CT lung images. This means that data augmentation helps to cover a wide variety of CT lung images to create a more consistent prediction. Similar to the single segmentation network, the self-supervised are able to predict output that is more closely related to the ground-truth when fed with CT lung images with different distribution as shown in the last row of the figure. The self-supervised multi SInfNet is also able to predict a better prediction on the consolidation on the CT lung images than the baseline models. Self-supervised learning also helps to improve the performance when the number of labels contain a small amount in the dataset. The labels on consolidation contains a smaller ratio when compared to the ground-glass opacities and non-infected region. Our self-supervised multi SInfNet was able to predict consolidation more accurately than the baseline multi SInfNet models.
   
 \ref{tab:multi-strongprior} Shows the result for the comparison between multiple segmentation InfNet. The prior fed into the multiple segmentation InfNet for this result is the ground truth prior obtained from the test set. For the result of this table, the prior is obtained from the test set. The prior is therefore the ground-truth of the segmentation of the single SInfNet. 
 We can see from the table that the muti self SInfNet is the best performing network compare to the rest of the network. We can see that data augmentation improves the consistency of the network. However, data augmentation does not neccessarily improve the performance of the network. Data augmentation makes a network more robust to outliers but does not necessarily improve the performance of the network in CT lung images. The self-supervised multi SInfNet improves the performance of the consolidation by a large amount especially in the jaccard metric. We can see the figure of the comparison between different multi segmentation SInfNet with strong prior (Priors that are obtained from the test dataset) in \ref{fig:multi-strongprior-comparison}.
 We can see that the self-supervised multi SInfNet does a better job at predicting consolidation - a label of a smaller amount than the rest of the labels in the dataset. Self-supervised learning helps to improve the performance of the network especially the labels that are of smaller amount compare to the rest of the labels. 


\subsection{Result for estimation of severity score}
\ref{tab:severity} shows the result of the severity score with different networks. We can see that having data augmentation improves the performance of the baseline InfNet in severity calculation. It achieves an F1 score of 0.61,  an amount of more than two times the performance of baseline InfNet network. Enabling a higher value of random cutout in the data augmentation seems to reduce the performance of the severity score calculation by a marginal amount. The result of baseline InfNet with data augmentation where the random cutout is set to 0.5 have an F1 score of 0.42 with severity calculation. As having a higher random cutout value improves the network capacity to predict the image inpainting of the CT lung images, the capacity for the network to generalize to other information is lost. Adding self-supervised learning to the baseline InfNet model improves the network performance to 0.55, the performance is still lower than the baseline InfNet with data augmentation but it is two times the performance of baseline InfNet. Our self-supervised InfNet with added data augmentation yields the best performing result. The self-supervised InfNet with added data augmentation achieves a result of 0.62. The combination of self-supervised learning and data augmentation helps the InfNet model to generalise to different tasks as both technique aids the model in learning detailed information about CT lung images and a more diverse CT lug images distribution.
\section{Conclusion}
Our results show that the integration of self-supervised image in-painting to the supervised multi segmentation InfNet improves the performance of both the ground-glass opacities and the consolidation in the dataset that we used. Additionally, adding focal loss and lookahead optimizer further improves our self-supervised multi InfNet and achieves 0.63 F1 scores. The improvement in the performance of the infected region segmentation for ground-glass opacities and consolidation of CT lung images can help prevent negatively assessing that a patient contains irregular patterns when the patients are healthy. This would prevent patients from receiving unnecessary treatments that could cause side effects. For the future work, we could apply this technique for other datasets that include segmenting Leukemia or breast cancer images.

\section*{Acknowledgment}
I would like to acknowledge Dr. PingZhao Hu and Dr. Carson Leung for the supervision of this Honours Project. I have learned a lot about the technique and best practices to write a good paper. I would like to thank Judah Zammit and Qian Liu too, they have recommended me several methods and tools to improve on the work for this project. I would also like to acknowledge Dr. Ruppa Thulasiram for the opportunity to be able to take this course with Dr. PingZhao Hu and Dr. Carson Leung as my supervisor. I would not be able to get this much opportunity and learned all these without them. I would like to add ICTCF and MedSeg as an acknowledgement for providing the publicly available COVID-19 CT lung images dataset. The publicly available dataset has helped us been able to carry out this research.
\begin{thebibliography}{1}
	%\textit{Basic Format for Books:}\\
	\bibitem{ref1} Alom, MZ., Rahman, MMS., Nasrin, MS., Taha, TM., Asari, VK. \textit{COVID-MTNet: COVID-19 Detection with Multi-Task Deep Learning Approaches.} arXiv:2004.03747, 2020.
	
	\bibitem{ref2} Yan, Q., Wang, B., Gong, D., et al. \textit{COVID-19 Chest CT Image Segmentation -- A Deep Convolutional Neural Network Solution.} arXiv:2004.10987, 2020.
	
	\bibitem{ref3} Ronneberger, O., Fischer, P., and Brox, T. \textit{U-net: Convolutional networks for biomedical image segmentation.} In MICCAI, pages 234–241. Springer, 2015. 2
	
	\bibitem{ref4} Kalluri, T., Varma, G., Chandraker, M., and Jawahar, CW. \textit{Universal semi-supervised semantic segmentation.} CoRR, abs/1811.10323, 2018.
	
	\bibitem{ref5} Misra, I., and van der Maaten, L.\textit{ Self-supervised learning of pretext-invariant representations.} arXiv preprint arXiv:1912.01991, 2019.
	
	\bibitem{ref6} Chen, T., Kornblith, S., Norouzi, M., and Hinton, G. \textit{A simple framework for contrastive learning of visual representations.} arXiv:2002.05709, 2020.
	
	\bibitem{ref7} Newell, A., Deng, J. \textit{How Useful is Self-Supervised Pretraining for Visual Tasks?} arXiv:2003.14323, 2020.
	
	\bibitem{ref8} Novosel, J., Viswanath, P., and Arsenali, B. \textit{Boosting Semantic Segmentation With Multi-Task Self-Supervised Learning for Autonomous Driving Applications.} In Proc. of NeurIPS - Workshops, pages 1–11, Vancouver, BC, Canada, Dec. 2019.
	
	\bibitem{ref9} Kahl, F. \textit{“Fine-grained segmentation networks: Self-supervised segmentation for improved long-term visual localization,”} in Proceedings of the IEEE International Conference on Computer Vision, 2019, pp. 31–41.
	
	\bibitem{ref10} Chang, YC., Yu, CJ., Chang, SC., et al. \textit{Pulmonary sequelae in convalescent patients after severe acute respiratory syndrome: evaluation with thin-section CT.} Radiology 2005; 236(3):1067-1075.
	
	\bibitem{ref11} Yang, R., Li, X., Liu, H., Zhen, Y., Zhang, X., Xiong, Q., et al. \textit{Chest CT Severity Score: An Imaging Tool for Assessing Severe COVID-19. Radiol Cardiothorac Imaging.} 2020;2(2):e200047.
	
	\bibitem{ref12} Shan, F., Gao, Y., Wang, J., Shi, W., Shi, N., Han, M., Xue, Z., and Shi, Y. \textit{Lung Infection Quantification of COVID-19 in CT Images with Deep Learning.} arXiv preprint arXiv:2003.04655, 1-19, 2020.

	\bibitem{ref13} Yan, Q., Wang, B., Gong D., et al. \textit{COVID-19 Chest CT Image Segmentation – A Deep Convolutional Neural Network Solution.} arXiv preprint arXiv:2004.10987, 2020.

	\bibitem{ref14} Fan, DP., Zhou, T., Ji, GP., et al. \textit{Inf-Net: Automatic COVID-19 Lung Infection Segmentation from CT Scans.} arXiv preprint arXiv:2004.14133v2, 2020.

	\bibitem{ref15} Alexander Kolesnikov, Xiaohua Zhai, and Lucas Beyer. \textit{Revisiting self-supervised visual representation learning.} In Conference on Computer Vision and Pattern Recognition (CVPR), 2019.

	\bibitem{ref16} Trinh, TH., Luong, MT., and Le, QV. \textit{Selfie: Self-supervised pretraining for image	embedding.} arXiv preprint arXiv:1906.02940, 2019.

	\bibitem{ref17} Frinken, V., Zamora-Martinez, F., Espana-Boquera, S., Castro-Bleda, M. J., Fischer, A., and Bunke, H. (2012). \textit{Long-short term memory neural networks language modeling for handwriting recognition.} In Pattern Recognition (ICPR), 2012 21st International Conference on, pages 701–704. IEEE.

	\bibitem{ref18}  LeCun, Y., Haffner, P., Bottou, L., and Bengio, Y. \textit{Object recognition with gradient-based learning.} In Shape, contour and grouping in computer vision, pages 319–345. 1999.

	\bibitem{ref19} Kingma, Diederik, P. and Welling, M. \textit{Auto-Encoding Variational Bayes.} In The 2nd International Conference on Learning Representations (ICLR), 2013.

	\bibitem{ref20}  Goodfellow IJ., Pouget-Abadie, J., Mirza, M., Xu, B., Warde-Farley, D., Ozair, S., Courville, AC., and Bengio, Y. \textit{Generative adversarial nets.} In Proceedings of NIPS, pages 2672– 2680, 2014.

	\bibitem{ref21} Zhao, JY., Zhang, YC., He, XH., Xie, PT. \textit{COVID-CT-Dataset: a CT scan dataset about COVID-19.} arXiv preprint arXiv: 2003.13865, 2020.

	\bibitem{ref22} Cohen, JP., Morrison, P., and Dao, L.  \textit{COVID-19 Image Data Collection.} arXiv preprint arXiv: 2003.11597, 2020. https://github.com/ieee8023/covid-chestxray-dataset.

	\bibitem{ref23} Ning, Lei, WS., Yang SJ., et al. (2020). \textit{iCTCF: an integrative resource of chest computed tomography images and clinical features of patients with COVID-19 pneumonia.} 10.21203/rs.3.rs-21834/v1.

	\bibitem{ref24} Zhang, K., Liu, XH., Shen, J., et al. \textit{Clinically Applicable AI System for Accurate Diagnosis, Quantitative Measurements and Prognosis of COVID-19 Pneumonia Using Computed Tomography.} DOI: 10.1016/j.cell.2020.04.045.
	
	\bibitem{ref25}  Singh, S., Batra, A., Pang, G., Torresani, L., Basu, S.,  Paluri, M., and Jawahar, C. V.  \textit{Self-supervised feature learning for semantic segmentation of
	overhead imagery.} In BMVC, 2018.

	\bibitem{ref26} \textit{COVID-19 CT segmentation dataset}. Retrieved from http://medicalsegmentation.com/covid19/.

\end{thebibliography}

\end{document}