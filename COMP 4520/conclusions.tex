\section{Conclusion}
Our results show that the integration of self-supervised image in-painting to the supervised multi segmentation InfNet improves the performance of both the ground-glass opacities and the consolidation in the dataset that we used. Additionally, adding focal loss and lookahead optimizer further improves our self-supervised multi InfNet and achieves 0.63 F1 scores. The improvement in the performance of the infected region segmentation for ground-glass opacities and consolidation of CT lung images can help prevent negatively assessing that a patient contains irregular patterns when the patients are healthy. This would prevent patients from receiving unnecessary treatments that could cause side effects. For the future work, we could apply this technique for other datasets that include segmenting Leukemia or breast cancer images.

\section*{Acknowledgment}
I would like to acknowledge Dr. PingZhao Hu and Dr. Carson Leung for the supervision of this Honours Project. I have learned a lot about the technique and best practices to write a good paper. I would like to thank Judah Zammit and Qian Liu too, they have recommended me several methods and tools to improve on the work for this project. I would also like to acknowledge Dr. Ruppa Thulasiram for the opportunity to be able to take this course with Dr. PingZhao Hu and Dr. Carson Leung as my supervisor. I would not be able to get this much opportunity and learned all these without them. I would like to add ICTCF and MedSeg as an acknowledgement for providing the publicly available COVID-19 CT lung images dataset. The publicly available dataset has helped us been able to carry out this research.